\documentclass[a4paper,11pt]{article}
\usepackage[utf8]{inputenc}
\usepackage{geometry}
\geometry{top=2.5cm, bottom=2.5cm, left=2.5cm, right=2.5cm}
\usepackage{hyperref}
\usepackage{enumitem}
\usepackage{longtable}

\title{Curriculum Vitae}
\author{Nuno David Lopes}
\date{}

\begin{document}

\maketitle
\section{Long version}
Nuno David Lopes is a highly experienced academic professional specializing in numerical analysis, applied mathematics, and industrial mathematics. Holding a PhD in Mathematics, he has an extensive background in both teaching and research. He currently serves as a professor at the Instituto Superior de Engenharia de Lisboa (ISEL), a part of the Instituto Politécnico de Lisboa, where he teaches various mathematical disciplines including Numerical Analysis, Linear Algebra, and Mathematical Modelling. His expertise extends to supervising internships and providing guidance for students working on projects related to the application of mathematics in industrial contexts.

Nuno has been involved in multiple successful collaborations with industry partners, including the development of mathematical models for various sectors such as engineering, energy, and infrastructure. These partnerships have allowed him to integrate theoretical knowledge with practical applications, especially in the fields of fluid dynamics, oceanography, and structural engineering. Notably, he has supervised and co-supervised several master’s theses and internship projects at institutions such as SOLVIT, LNEC (Laboratório Nacional de Engenharia Civil), and GALP, contributing to the advancement of research in areas like predictive modeling and the application of artificial intelligence in infrastructure management.

His research focuses on computational and analytical methods for solving complex mathematical problems. Nuno has authored and co-authored several textbooks and research papers, including contributions to topics like optimization, numerical simulation, and the use of machine learning in mathematical modeling. He is also passionate about integrating technology into education and research, often collaborating on projects that promote the use of computational tools like Python and Julia in academic settings.

In addition to his teaching and research roles, Nuno has taken on various administrative and leadership responsibilities. He has served as a member of several academic committees and has been actively involved in shaping the curriculum for the Mathematics Applied to Technology and Business (LMATE) program. His role as a member of the coordinating committee for LMATE and as the head of the LAB4MAT computational laboratory has helped him influence the academic structure and foster innovation in the educational environment. 

Throughout his career, Nuno has contributed to the promotion of scientific knowledge through lectures, seminars, and public outreach activities. His talks on the applications of mathematics in real-world problems have been presented at various conferences and institutions, underscoring his commitment to bridging the gap between academia and industry.

\section{Short version-1}

I am Nuno David Lopes, a professor and researcher in the field of Mathematics, specializing in Numerical Analysis. With a PhD in Mathematics and experience in academia, my work focuses on applying mathematical techniques to real-world problems, particularly in mathematical modeling, computational methods, and the application of neural networks.

I have worked on diverse research and industry-related projects, including numerical modeling for oceanography, structural engineering, and optimization in energy systems. My collaborations with institutions like LNEC, SOLVIT, GALP, WURTH, Sandometal, and CDRSP-IPLeiria have involved solving complex mathematical problems and developing computational solutions.

As an educator, I teach a wide range of mathematical courses, focusing on practical applications and computational techniques. I am dedicated to mentoring students, enhancing their critical thinking and problem-solving skills, and guiding them in applied mathematics, particularly through internships and research projects that connect academia with industry.

I am open to new opportunities for collaboration, applying my expertise to solving complex real-world challenges using advanced mathematical techniques.

\section{Short version-2}
I’m Nuno David Lopes, an Adjunct Professor at the Departamento de Matemática, Instituto Superior de Engenharia de Lisboa, Instituto Politécnico de Lisboa. My expertise lies in Numerical Analysis, with a particular focus on Computational Fluid Dynamics and modeling of water waves.

In recent years, I have expanded my research into Industrial Mathematics, particularly through supervising internships and collaborative projects with several Portuguese companies and institutions.

Throughout my career, I have worked on various research and industry-related projects, including numerical modeling for oceanography, structural engineering, and energy systems optimization. During my PhD, I developed DOLFWAVE (C++ and Python), a FEniCS-based library designed to solve surface water waves problems.

As an educator, I teach a wide range of mathematical courses, emphasizing practical applications and computational techniques. I am dedicated to mentoring students, helping them develop their critical thinking, problem-solving skills, and guiding them in applying mathematics to real-world challenges, especially through internships and collaborative research projects.

I am always open to new opportunities to collaborate and apply my expertise to tackle complex challenges, particularly those that require advanced mathematical techniques.

\end{document}
