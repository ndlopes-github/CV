%% Copyright (C) 2024 Nuno David Lopes.-----------------------------------
%%
%% e-mail: nuno.lopes@isel.pt
%%
%% First added:  2014-05-02
%% Last changed: 2024-06-18


\documentclass[10pt,a4paper,sans]{moderncv}

\moderncvstyle{classic}
% moderncv themes
%\moderncvstyle{casual}       % style options are
                              %'casual' (default), 'classic', 'oldstyle' and 'banking'


\moderncvcolor{blue} % color options are 'red', 'purple', 'grey' and 'black'
%\renewcommand{\familydefault}{\sfdefault}   % to set the default
                                             % font; use '\sfdefault'
                                             % for the default sans
                                             % serif font, '\rmdefault'
                                             % for the default roman
                                             % one, or any tex font name

%\nopagenumbers{}   % uncomment to suppress automatic page numbering for CVs longer than one page

\usepackage{lmodern}
\usepackage[portuguese]{babel}

\usepackage[utf8]{inputenc}
\usepackage[T1]{fontenc}


% adjust the page margins
\usepackage[scale=0.75]{geometry}
%\setlength{\hintscolumnwidth}{3cm}                % if you want to change the width of the column with the dates
%\setlength{\makecvtitlenamewidth}{10cm}           % for the 'classic' style, if you want to force the width allocated to your name and avoid line breaks. be careful though, the length is normally calculated to avoid any overlap with your personal info; use this at your own typographical risks...
\usepackage{lastpage}
\rfoot{\addressfont\itshape\textcolor{gray}{\thepage\ de \pageref{LastPage}}}

% personal data
\name{Nuno}{Lopes}

\title{Curriculum Vit\ae}  % optional, remove / comment the line if not wanted

%\address{street and number}{postcode city}{country}% optional, remove / comment the line if not wanted; the "postcode city" and "country" arguments can be omitted or provided empty
%\address{}{}{1700-018 Lisboa, (Portugal)}
\phone[mobile]{+351 916833519} % optional, remove / comment the line if
                               % not wanted; the optional "type" of the
                               % phone can be "mobile" (default), "fixed" or "fax"
\email{nuno.lopes@isel.pt}     % optional, remove / comment the line if not wanted
%\homepage{http://ptmat.fc.ul.pt/$\sim$ndl} % optional, remove / comment
%the line if not wanted
\extrainfo{ORCID: 0000-0001-9577-0347\\
CIÊNCIA ID: 3312-D2F4-EF2F}
%\extrainfo{https://github.com/ndlopes-github}
\quote{junho de 2024}


%----------------------------------------------------------------------------------
%            content
%----------------------------------------------------------------------------------
\begin{document}

\makecvtitle


\section{Dados Pessoais}
\cvline{}{Nome completo: Nuno David de Jesus Lopes}
\cvline{}{Data de Nascimento: 31 de dezembro de 1976}
\cvline{}{Nacionalidade: Portuguesa}


\section{Formação Académica}

\cvline{05/2014}{\textbf{Doutoramento em
    Matemática},~\emph{especialização em Análise Numérica}, Faculdade de
  Ciências e Tecnologia da Universidade Nova de Lisboa.\newline
Orientador: Prof. Doutor Pedro Jorge da Silva Pereira\newline
Co-orientador: Prof. Doutor Luís Trabucho de Campos\newline
Título da dissertação:~\emph{Métodos Analíticos e Numéricos do Tipo Elementos
Finitos Contínuos e Contínuos/Descontínuos para o Estudo de Modelos de
Boussinesq Melhorados para a Propagação de Ondas}}

\cvline{01/2004}{\textbf{Mestrado em Matemática},~\emph{especialização em
  Análise Funcional e Equações Diferenciais}, Faculdade de Ciências da Universidade de
  Lisboa.
% aprovado com a
%  classificação de Muito Bom.
\newline
  Orientador: Prof. Doutor Luís Trabucho de Campos\newline
Título da dissertação: \emph{Aplicação do Método dos Elementos Finitos
  na Análise de uma Classe de Problemas de Convecção-Difusão}}

\cvline{07/2000}{\textbf{Licenciatura em Matemática Pura}, Faculdade de
  Ciências da Universidade de Lisboa.
% aprovado com
%  classificação de treze valores.
}

\section{Actividades anteriores e situação actual}
\cvline{2014 -- presente}{Professor Adjunto no
  Instituto Superior de Engenharia de Lisboa, Instituto Politécnico de Lisboa.}
\cvline{2023 -- 2024}{Professor Adjunto Convidado na 
  Escola Superior de Tecnologia do Barreiro, Instituto Politécnico de Setúbal ($1^o$ semestre de 2023/24).}
  
\cvline{2004 -- 2014}{Professor Equiparado a Assistente do 2º Triénio no
  Instituto Superior de Engenharia de Lisboa, Instituto Politécnico de Lisboa.}
\cvline{2002 -- 2004}{Professor Equiparado a Assistente do 1º Triénio no
  Instituto Superior de Engenharia de Lisboa, Instituto Politécnico de
  Lisboa.}
\cvline{2002 -- 2002}{Professor Equiparado a Assistente do 1º Triénio na
  Escola Superior de Tecnologia e Gestão do Instituto Politécnico de
  Bragança.}
\cvline{2001 -- 2002}{Docente contratado na Faculdade de Ciências e
  Tecnologia da Universidade Nova de Lisboa.}


\subsection{Bolsas de estudo e outras}
\cvline{2007 -- 2010}{Bolsa de Doutoramento do Instituto
  Superior de Engenharia de Lisboa, Instituto Politécnico de Lisboa, de
  fevereiro de 2007 a fevereiro de 2010.}
\cvline{2000 -- 2001}{Bolsa de Iniciação à Investigação do Departamento de
  Matemática da Faculdade de Ciências da Universidade de Lisboa, no ano
  lectivo de 2000/2001.}

\newpage
\section{Actividades de investigação e desenvolvimento}
\subsection{Interesses de investigação}
\cvline{}{Métodos analíticos e numéricos do tipo elementos finitos para
  o estudo de modelos de Mecânica
  dos Fluidos, dando especial ênfase ao desenvolvimento de modelos do tipo
  Boussinesq para a propagação de ondas marítimas de
  superfície. Recentemente tem desenvolvido investigação em vários temas
  da Matemática Industrial.}

\subsection{Inclusão em equipas de investigação}
\cvline{2022 -- presente}{Membro integrado da equipa de investigação do Centro de Matemática Computacional e Estocástica da Universidade de Lisboa.}
\cvline{2015 -- 2019}{Membro integrado da equipa de investigação do Centro de Matemática e Aplicações da Universidade Nova de Lisboa.}
\cvline{2009 -- 2015}{Membro integrado da equipa de investigação do Centro de Matemática e Aplicações Fundamentais
 da Universidade de Lisboa.}
\cvline{2014 -- 2015}{Membro colaborador da equipa de investigação do Centro de Matemática e Aplicações da Universidade Nova de Lisboa.}
\cvline{2007 -- 2008}{Membro colaborador da equipa de investigação do Centro de Matemática e Aplicações Fundamentais
 da Universidade de Lisboa.}

\subsection{Artigos em revistas de circulação internacional com
  arbitragem científica}
\cvline{2024}{H.~B.~Oliveira~e~N.~D.~Lopes, \emph{Continuous/Discontinuous Finite Element 
Approximation of a 2D Navier-Stokes Problem Arising in Fluid Confinement}, International Journal of Numerical Analysis and Modeling, Volume 21, Number 3, Pages 315–352, 2024, DOI: 10.4208/ijnam2024-1013 }
\cvline{2023}{R. Enguiça e N.~D.~Lopes, \emph{The modeling of urban races},
Journal of Mathematics in Industry, Springer, 2023,
\url{doi:10.1186/s13362-023-00136-3}}

\cvline{2021}{N.~D.~Lopes~e~H.~B.~Oliveira, \emph{Continuous/discontinuous
Galerkin approximations for a fourth-order nonlinear problem}, Computers
and Mathematics with Application,~\textbf{97}, pages 122--152,Elsevier,
{2021}.
\url{doi:10.1016/j.camwa.2021.05.034}}.% Factor de impacto: 3,476.}

\cvline{2015}{P.~J.~S.~Pereira,~N.~D.~Lopes,~e~L.~Trabucho,
\emph{Soliton-type and other travelling wave solutions for
an improved class of non-linear sixth-order Boussinesq equations},
Nonlinear Dynamics, Springer, 2015. \url{doi:10.1007/s11071-015-2196-9.}}
%Factor de impacto: 2,849.}

\cvline{2012}{N.~D.~Lopes,~P.~J.~S.~Pereira~e~L.~Trabucho, \emph{A numerical
  analysis of a class of generalized Boussinesq-type equations using
  continuous/discontinuous FEM}, International Journal for Numerical
  Methods in Fluids,~\textbf{69} (7), pages 1186--1218, John Wiley $\&$
  Sons, Ltd, 2012. \url{doi:10.1002/fld.2631.}}% Factor de impacto: 1,244.}

\subsection{Capítulos de livros}

\cvline{2012}{N.~D.~Lopes,~P.~J.~S.~Pereira~e~L.~Trabucho, \emph{Improved
  Boussinesq Equations for Surface Water Waves}, Automated Solution of
  Differential Equations by the Finite Element Method, Eds. A. Logg,
  K.-A. Mardal and G. N. Wells, LNCSE Volume 84, pages 471--504,
  Springer 2012. \url{doi:10.1007/978-3-642-23099-8\_25.}}

\subsection{Livro técnico-científico}
\cvline{2019}{F.~Correia dos Santos, Jorge Duarte, N. D.~Lopes,
  \emph{Fundamentos de Análise Numérica: com Python e R},
  2$^{\underline{a}}$ ed., Edições Sílabo,  2019}






\subsection{Publicações em actas de encontros científicos}
\cvline{2023}{J.~E.~Marxen, T.~Charters, S.~Lopes; N.~D.~Lopes, N.~Fernandes Alves e P.~Pascoal-Faria, \emph{An introduction to scaffold design using topology optimization methods}, AIP Conference Proceedings, 2849 (1), \url{doi: 10.1063/5.0162364}}
\cvline{2023}{L.~Pinheiro, N.~Morgado, A.~Gomes, S.~Lopes, N.~D.~Lopes,
  A.~Prior, J.~Fortes, \emph{Uso de Redes Neuronais para Melhoria das
    Previsões de um Sistema de Alerta Para Riscos em Portos},
   16$^o$ Congresso da água, Laboratório Nacional de Engenharia
  Civil, Lisboa, Portugal, 2023.}

\cvline{2023}{A.~Mendonça, B.~Machado, M.~T.~Reis, J.~Santos, C.~Fortes,
  N.~D.~Lopes, A.~Prior, \emph{Evolução do Dano do Manto Protetor do Quebramar de Talude do Porto da Ericeira: Análise Probabilística},
   16$^o$ Congresso da água, Laboratório Nacional de Engenharia
  Civil, Lisboa, Portugal, 2023.}

\cvline{2023}{L.~Pinheiro, N.~Morgado, A.~Gomes, S.~Lopes, N.~D.~Lopes,
  A.~Prior, J.~Fortes, \emph{Neural networks for optimization of an early warning system for moored ships in harbours},
   ICCE 2022, 37th International Conference on Coastal Engineering,
   December, 2022. \url{doi:10.9753/icce.v37.papers.60}}

\cvline{2012}{N.~D.~Lopes,~P.~J.~S.~Pereira~e~L.~Trabucho, \emph{A
    numerical analysis of a class of KdV-BBM equations using
    Continuous/Discontinuous Galerkin Finite Element Method},
  Proceedings of MEFTE 2012, IV Conferência Nacional em Mecânica dos
  Fluidos, Termodinâmica e Energia, Laboratório Nacional de Engenharia
  Civil, Lisboa, Portugal, 2012.}

\cvline{2011}{N.~D.~Lopes,~P.~J.~S.~Pereira~e~L.~Trabucho, \emph{Um método
  numérico para uma classe de equações de Boussinesq para ondas de
  superfície com batimetria dependente do tempo}, Boletim da SPM--Encontro
  Nacional da SPM 2010, páginas 42-46, 2011.}

\cvline{2010}{N.~D.~Lopes,~P.~J.~S.~Pereira~e~L.~Trabucho, \emph{A C/DG-FEM
  Solution of an Improved Boussinesq System for Surface Water Waves},
  Proceedings of the Fifth European Conference on CFD, ECCOMAS CFD 2010,
  Laboratório Nacional de Engenharia Civil, Lisboa, Portugal, 2010.}

\cvline{2004}{N.~D.~Lopes~e~L.~Trabucho, \emph{Aplicação do Método dos
  Elementos Finitos na Análise de uma Classe de Problemas
  Convecção-Difusão}, Congresso de Métodos Computacionais em Engenharia,
  Laboratório Nacional de Engenharia Civil, Lisboa, Portugal, 2004.}

\subsection{Desenvolvimento de Software}
\cvline{2010 -- 2015}{\emph{DOLFWAVE: a library for surface water waves
    problems}, \textrm{http://launchpad.net/dolfwave}, com supervisão
  científica de~P.~J.~S.~Pereira~ e~L.~Trabucho.
  }


\subsection{Comissões  organizadoras de conferências}
%% \cvline{2020}{Membro da
%%   comissão organizadora local  da conferência nacional: \emph{Workshop FromIndustry -- preESGI 2020}
%%   organização conjunta com a rede
%% Portuguesa de Matemática para a Indústria e Inovação -- PT-MATHS-IN,
%% Instituto Superior de Engenharia de Lisboa, 28 de abril de 2020,
%% Lisboa, Portugal.}
\cvline{2024}{Membro da
  comissão organizadora  da conferência internacional: \emph{amim'24:
    autumn meeting for industrial mathemathics},
  para decorrer a 4 e 5 de outubro de 2024, Foz do Arelho, Portugal.}

\cvline{2023}{Membro da
  comissão organizadora  da conferência nacional: \emph{MATH'23:
    Networking Workshop},
Instituto Superior de Engenharia de Lisboa, 11 e  12 maio de 2023, Lisboa, Portugal.}


\cvline{2022}{Membro da
  comissão organizadora local  da conferência internacional : \emph{ESGI161}
  organização conjunta com a rede
Portuguesa de Matemática para a Indústria e Inovação -- PT-MATHS-IN,
Instituto Superior de Engenharia de Lisboa, 4 a 9 de setembro de 2022, Lisboa, Portugal.}

\cvline{2022}{Membro da
  comissão organizadora  da conferência nacional: \emph{LMATE'22:
    Networking Workshop}
  organização da comissão coordenadora da LMATE-ISEL,
Instituto Superior de Engenharia de Lisboa, 5 e  6 maio de 2022, Lisboa, Portugal.}


\cvline{2020}{Membro da
  comissão organizadora local  da conferência nacional: \emph{Workshop FromAcademy -- preESGI 2020}
  organização conjunta com a rede
Portuguesa de Matemática para a Indústria e Inovação -- PT-MATHS-IN,
Instituto Superior de Engenharia de Lisboa, 22 de janeiro de 2020, Lisboa, Portugal.}
\newpage
\subsection{Comissões  científicas de conferências}
\cvline{2024}{
  Membro da comissão científica da conferência internacional:
  \emph{ESGI181}, organização pela rede Portuguesa de Matemática para a
  Indústria e Inovação -- PT-MATHS-IN, 8 a 12 de julho de 2024, UTAD, Vila Real, Portugal.}
\cvline{2023}{
  Membro da comissão científica da conferência internacional:
  \emph{ESGI174}, organização pela rede Portuguesa de Matemática para a
  Indústria e Inovação -- PT-MATHS-IN, 3 a 7 de julho de 2023, ESTG-IPP, Felgueiras, Portugal.}

\cvline{2018} {Membro do comité científico da conferência internacional: \emph{Mathematics for Smart Security: Challenges and
  Opportunities}, organização pela rede
Portuguesa de Matemática para a Indústria e Inovação -- PT-MATHS-IN,
Instituto Superior de Engenharia de Lisboa, 19 de outubro de 2018, Lisboa, Portugal.}

\subsection{Comunicações orais em congressos e seminários}
\cvline{2024}{N.~D.~Lopes, "CD-FEM solutions of fourth-order nonlinear models for the confinement of fluid flows", International Conference Mathematical Analysis and Applications in Science and Engineering: ICMASC24, 20 a 22 de junho, Instituto Superior de Engenharia do Porto, Porto, Portugal.}

\cvline{2022}{N.~D.~Lopes, ``Alguns métodos e resultados numéricos sobre
  o confinamento de um fluido, a injecção assistida por água num molde e
  corridas urbanas'', Seminário de Física e Matemática, Departamento de
  Física, Instituto Superior de Engenharia de Lisboa, 2 de maio de 2022.}
\cvline{2021}{N.~D.~Lopes, ``C/DG-FEM solutions of a
  fourth-order nonlinear model for the confinement of fluid
  flows'',trabalho conjunto com H.~B.~de~Oliveira, Encontro Nacional da
  Sociedade Portuguesa de Matemática 2021, july 16, 2021.}

\cvline{2020}{E.~C.~Figueiredo (orador), N.~D.~Lopes, Ricardo Enguiça, Tiago
  Charters de Azevedo, Rui Silva, Bruno Mendes, Artur Mateus, Nuno Alves and Paula Pascoal-Faria,
  ``MathMould I\&D Project: Analytical and Numerical Methods in Water
  Assisted Injection in Multi--Tubular Moulds'',
  poster com  apresentação oral por E.~C.~Figueiredo em RESIM --
  Research on Susteinable and Intelligent Manufacturing, International
  Virtual Conference, Politécnico de Leiria e Universitat Internacional
  de Catalunya, 4-5 June, 2020.}

\cvline{2017}{N.~D.~Lopes,
  ``Analytical and Numerical Methods for some Improved
  Boussinesq Models for Surface Water Waves'', trabalho conjunto com
  ~P.~J.~S.~Pereira~e~L.~Trabucho, Seminário de Análise do
  Centro de Matemática e Aplicações do Departamento de Matemática da
  Faculdade de Ciências e Tecnologia da Universidade Nova de Lisboa, 15 de março de 2017, Almada, Portugal.}

\cvline{2015}{N.~D.~Lopes,
  ``Analytical and Numerical Methods of the type FEM-C/D for Improved
  Boussinesq Models'', trabalho conjunto com
  ~P.~J.~S.~Pereira~e~L.~Trabucho, Seminário de Matemática Aplicada e Análise
  Numérica  do Centro de Matemática e Aplicações do Instituto Superior
  Técnico, 16 de julho de 2015, Lisboa, Portugal.}


\cvline{2012}{N.~D.~Lopes, ``FEM
  solutions of some Boussinesq-type models for surface water waves'', trabalho conjunto com ~P.~J.~S.~Pereira~e~L.~Trabucho,
  Seminários do Centro de Matemática da Universidade de Coimbra, 11 de
  julho de 2012, Coimbra, Portugal.}

\cvline{2012}{N.~D.~Lopes, ``A numerical
  analysis of a class of KdV-BBM equations using
  Continuous/Discontinuous Galerkin Finite Element Method'',
  trabalho conjunto com ~P.~J.~S.~Pereira~e~L.~Trabucho, MEFTE 2012,
  IV Conferência Nacional em Mecânica dos Fluidos, Termodinâmica e
  Energia, 28 e 29 de maio de 2012, Laboratório Nacional de Engenharia
  de Lisboa, Lisboa, Portugal.}


\cvline{2011}{N.~D.~Lopes, ``Solutions
  of some improved Boussinesq-type models for surface water waves using
  finite element methods'', trabalho conjunto com
  ~P.~J.~S.~Pereira~e~L.~Trabucho, Recent trends in hyperbolic and
  related  PDEs: Theory, numerics and applications,
  21 de outubro de 2011, Instituto para a Investigação
  Interdisciplinar, Universidade de Lisboa,
  Lisboa, Portugal.}

\cvline{2011}{N.~D.~Lopes, ``A C/DG-FEM
  solution of an improved fourth-order Boussinesq model for surface
  water waves'', trabalho conjunto com ~P.~J.~S.~Pereira~e~L.~Trabucho,
  First European Meeting of PhD Students in Mathematics,
  28 e 29 de junho de 2011, LAMFA, Université de Picardie Jules Verne,
  Amiens, França.}

\cvline{2011}{N.~D.~Lopes, ``DOLFWAVE: a
  FEniCS Application for Water Waves Simulation'', trabalho conjunto com
  ~P.~J.~S.~Pereira~e~L.~Trabucho, SIAM Conference on
  Mathematical and Computational Issues in the Geosciences, 21 a 24 de
  março de 2011, Long Beach, Estados Unidos da América.}

\cvline{2010}{N.~D.~Lopes, ``Um método
  numérico para uma classe de equações de Boussinesq para ondas de
  superfície com batimetria dependente do tempo'', trabalho conjunto com
  ~P.~J.~S.~Pereira~e~L.~Trabucho, Encontro Nacional da
  SPM 2010, 8 a 10 de julho de 2010, Leiria, Portugal.}

\cvline{2010}{N.~D.~Lopes, ``A C/DG-FEM
  Solution of an Improved Boussinesq System for Surface Water Waves'',
  trabalho conjunto com ~P.~J.~S.~Pereira~e~L.~Trabucho,
  Fifth European Conference on CFD, ECCOMAS CFD 2010, 14 a 17 de junho
  de 2010, Laboratório Nacional de Engenharia de Lisboa, Lisboa,
  Portugal.}

\cvline{2010}{N.~D.~Lopes, ``DOLFWAVE - Use of FEniCS for
  surface waves modelling'', trabalho conjunto com ~P.~J.~S.~Pereira~e~L.~Trabucho, CBC Workshop on Tsunami Modelling,
  3 e 4 de junho de 2010, Center of Biomedical Computing,
  Simula Research Laboratory, Oslo, Noruega.}

\cvline{2009}{N.~D.~Lopes, ``Aplicações
  do Método dos Elementos Finitos a
  uma Classe de Sistemas de Boussinesq'', trabalho conjunto com
  ~P.~J.~S.~Pereira~e~L.~Trabucho, seminário do
  Centro de Matemática e Aplicações, 21 de janeiro de 2009,
  Universidade Nova de Lisboa, Monte de Caparica, Portugal.}

\cvline{2004}{N.~D.~Lopes, "Aplicação do Método dos
  Elementos Finitos na Análise de uma Classe de Problemas
  Convecção-Difusão", trabalho conjunto com L.~Trabucho, Congresso de Métodos Computacionais em Engenharia,
  31 de maio a 2 de junho de 2004,
  Laboratório Nacional de Engenharia Civil, Lisboa, Portugal.}

\cvline{2003}{N.~D.~Lopes, "Análise Numérica de um
  Modelo de Convecção-Difusão: Aplicação ao Estuário do Tejo", trabalho
  conjunto com L.~Trabucho,
  Seminários de Análise Funcional, Equações Diferenciais e Análise
  Numérica, 6 de fevereiro de 2003,
  Centro de Matemática e Aplicações Fundamentais da
  Universidade de Lisboa,  Lisboa, Portugal.}

\cvline{2001}{N.~D.~Lopes, "O Método dos Elementos Finitos numa equação
  de Convecção-Difusão", trabalho conjunto com L.~Trabucho,
  Seminários de Análise Funcional, Equações Diferenciais e Análise
  Numérica, 20 de dezembro de 2001,
  Centro de Matemática e Aplicações Fundamentais da
  Universidade de Lisboa, Lisboa, Portugal.}

%\subsection{Artigos em prepara��o para publica��o}

%\cvline{}{P.~J.~S.~Pereira,~N.~D.~Lopes~e~L.~Trabucho, \emph{Existence
%    and non-existence of some soliton-type solutions of an improved class
%    of higher order Boussinesq equations}.}

%\cvline{}{N.~D.~Lopes,~P.~J.~S.~Pereira~e~L.~Trabucho,
%  \emph{A continuous/discontinuous finite element method
%    applied to a parameterised class of KdV-BBM equations.}}


\subsection{Participação nos \emph{European Study Groups with Industry}}
\cvline{2024}{Colaborador no grupo: \emph{Enhancing Wine Sensory Evaluation: Developing a Robust System for Consistent and Accurate Taster Scores}, problema
  apresentado por Comissão de Viticultura da Região dos Vinhos Verdes em \textit{181th
    European Study Group with Industry}, 2024, UTAD, Vila Real, Portugal.}

\cvline{2023}{Colaborador no grupo: \emph{Optimization of Sole Production
Scheduling}, problema
  apresentado por SoftIdea em \textit{174th
    European Study Group with Industry}, 2023, ESTG-IPP, Felgueiras, Portugal.}

\cvline{2022}{Colaborador no grupo: \emph{Optimal Site Positioning}, problema
  apresentado por Solvit em \textit{161th
    European Study Group with Industry}, 2022, ISEL-IPL, Lisboa, Portugal.}

\cvline{2019}{Colaborador no grupo: \emph{Vipex Container Lid}, problema
  apresentado por Vipex em \textit{155th
European Study Group with Industry}, 2019, Escola Superior de Saúde do Instituto Politécnico de Leiria, Leiria, Portugal.}


\cvline{2018}{Colaborador no grupo: \emph{Managing start waves for mass
  running events}, problema apresentado por Lap2Go em  \textit{140th
European Study Group with Industry}, 2018, Escola de
Tecnologia do Instituto Politécnico de Setúbal, Barreiro, Portugal.}

\cvline{2008} {Colaborador no grupo: \emph{Cooling of a rotor}, problema apresentado por Biosafe em  \textit{65th
  European Study Group with Industry}, Faculdade de Engenharia da
  Universidade do Porto, 2008, Porto, Portugal}

\cvline{2007}{Colaborador no grupo: \emph{Brisa}, problema apresentado
  por Brisa em   \textit{60th
European Study Group with Industry}, ISEL, 2007, Lisboa, Portugal}


\subsection{Relatórios técnico-científicos}
\cvline{2024\\ ESGI174}{A. Moura, I. Lopes, J. Matias, N. Lopes e R. Enguiça,
  \emph{Optimization of Sole Production
Schedulin}, problem presented by SoftIdea at the
  174th European Study Group with Industry (2023), Cambridge Open Engage,
  Mathematics in Industry Reporst, 2024, DOI:{10.33774/miir-2024-zhmg5}}

\cvline{2022\\ ESGI161}{A. Araújo, J. O. Cerdeira, N. Lopes, A. Moura e E. Silva,
  \emph{Optimal site positioning}, problem presented by Solvit at the
  161th European Study Group with Industry (2022), Cambridge Open Engage,
  Mathematics in Industry Reporst, 2023, DOI:{10.33774/miir-2023-tkf3c}}

\cvline{2019\\ ESGI155}{N.~Ferreira, N.~Alves, N.~Lopes, P.~Faria e R. Barreira:
  \emph{Vipex Container Lid}, problem presented by Vipex at the  155th
  European Study Group with Industry, reports of the Portuguese Study
  Groups, 155th European Study Group with Industry, 1 to 5 of  July,
  2019, School of Health, Polytechnic Institute of Leiria, Leiria, Portugal.}

\cvline{2018\\ ESGI140}{P.~Amaral, S.~Barbeiro, R.~Barreira, M.~Cruz,
  R.~Enguiça, N.~Lopes, M.~McPhail e F. Wechsung: \emph{Managing start waves for mass
  running events}, problem presented by Lap2Go,  reports of the
Portuguese Study Groups, 140th European Study Group with Industry, 4 to 8 of  June, 2018, Barreiro
School of Technology, Polytechnic Institute of Setúbal,
Barreiro, Portugal}

\cvline{2015\\ \textbf{W{\"{u}}rth Portugal}}{N.~D.~Lopes, \emph{Profitability Optimization
    in Supply Chain: Problem presented by W{\"{u}}rth Portugal to
área Departamental de Matem\'{a}tica,
Instituto Superior de Engenharia de Lisboa}, novembro de 2015.}

\cvline{2008\\ ESGI65}{I.~Cruz, P.~Freitas~e~N.~D.~Lopes, \emph{Cooling of a
    rotor}, problem presented by BioSafe, reports of the Portuguese
  Study Groups, 65th European Study Group with Industry, 21 to 24 of
  April, 2008, Centro de Matemática da Universidade do Porto, Portugal.}


\subsection{Projectos científicos}
\cvline{2022 -- 2023}{Membro de equipa com função de investigador no projecto \emph{3A},
  projecto de Investigação, Desenvolvimento, Inovação e Criação
  Artística do Instituto Politécnico de Lisboa (IDI$\&$CA),   ref.:
  IPL/2022/3A\_ISEL.}

\cvline{2019 -- 2020}{Coordenador do projecto \emph{MathMould},
  projecto de Investigação, Desenvolvimento, Inovação e Criação
  Artística do Instituto Politécnico de Lisboa (IDI$\&$CA),   ref.:
  IPL/2019/MathMould\_ISEL.}

\cvline{2013 -- 2014}{Membro de equipa com função de Investigador no projecto PEst-OE/MAT/UI0209/2013, Projecto Estratégico - UI 209 - 2013-2014.}

\cvline{2011 -- 2012}{Membro de equipa com função de Investigador no projecto PEst-OE/MAT/UI0209/2011, Projecto Estratégico - UI 209 - 2011-2012.}

\cvline{2005 -- 2009}{Membro de equipa com função de investigador no projecto POCI/MAT/60587/2004, Análise Assimptótica Aplicada à Mecânica dos Meios Contínuos, FCT.}

\subsection{Projectos de cooperação com a comunidade}
\cvline{2015}{Assessor técnico-científico do
  \emph{Projeto de Otimização de Rentabilidade} associado ao
  \emph{Contrato de Prestação de Serviços do ISEL à W{\"{u}}rth} Portugal, setembro
  a dezembro de 2015.}
\newpage 
\subsection{Orientações de estágios no contexto da Matemática Industrial}
\cvline{2024\\ \textbf{SOLVIT}}{Inês Henriques, ``Railway Signal Monitoring An Algorithmic Approach for Improved Maintenance Strategies'', instituição de
 acolhimento: SOLVIT, classificação de 18 valores, estágio e dissertação no Mestrado em Matemática Aplicada à Indústria (MMAI),
 orientação conjunta com Eng.$^{\underline{\mbox{o}}}$ Nuno Cota (\textbf{SOLVIT}),
 Prof. Filipe Cal (ISEL), novembro de 2024,
 Departamento de Matemàtica do Instituto Superior de Engenharia de
 Lisboa.}
\cvline{2023\\ \textbf{LNEC}}{Bernardo Machado, ``Modelação Numérica em Quebra-Mares de
Talude e Estruturas de Proteção Marginal.
Casos de Estudo: Porto de pesca de Sines e
Porto da Praia da Vitória'', instituição de
 acolhimento: LNEC Portugal, classificação de 18 valores,  estágio curricular da Licenciatura em
 Matemática Aplicada à  Tecnologia e à Empresa (LMATE),
 orientação conjunta com Eng.$^{\underline{\mbox{a}}}$ Ana Mendonça (\textbf{LNEC}),
 Prof.$^{\underline{\mbox{a}}}$ Ana Filipa Prior (ISEL), setembro de 2023,
 Departamento de Matemàtica do Instituto Superior de Engenharia de
 Lisboa.}

\cvline{2022\\ \textbf{LNEC}}{Nuno Moreira Morgado
 , ``Calibração do Modelo Numérico SWAN com base em Redes Neuronais'', instituição de
 acolhimento: LNEC Portugal, classificação de 16 valores,  estágio curricular da Licenciatura em
 Matemática Aplicada à  Tecnologia e à Empresa (LMATE),
 orientação conjunta com Eng.$^{\underline{\mbox{a}}}$ Liliana Pinheiro (\textbf{LNEC}),
 Prof.$^{\underline{\mbox{a}}}$ Ana Filipa Prior (ISEL), julho de 2022,
 Departamento de Matemàtica do Instituto Superior de Engenharia de
 Lisboa.}

\cvline{2022\\ \textbf{SANDO-\\METAL}}{Inês Morais
 , ``Modelo de Comportamento Acústico em
Unidades de Tratamento de Ar'', instituição de
 acolhimento: Sandometal, classificação de 16 valores,  estágio curricular da Licenciatura em
 Matemática Aplicada à  Tecnologia e à Empresa (LMATE),
 orientação conjunta com Eng.$^{\underline{\mbox{o}}}$ Rui Santos,
 (\textbf{Sandometal}), Eng.$^{\underline{\mbox{o}}}$ Pedro Silva (\textbf{Sandometal})
 Prof.  Sérgio Lopes (ISEL), março de 2022,  Departamento de Matemática do Instituto Superior de Engenharia de Lisboa.}

\cvline{2021\\ \textbf{CDRSP-IPLeiria}}{
Jan Marxen, ``An Introduction to Scaffold Design using Topology Optimization Methods'', \emph{CDRSP}, classificação de 20 valores,  estágio curricular da Licenciatura em Matemática Aplicada à
  Tecnologia e à Empresa (LMATE),
  orientação conjunta com Prof.$^{\underline{\mbox{a}}}$ Doutora Paula Pascoal-Faria
  Faria (CDRSP),
  Prof. Doutor Sérgio Lopes (ISEL),
  Prof. Doutor Tiago Charters de Azevedo (ISEL),   julho
  de 2021, Departamento de Matemática do Instituto Superior de Engenharia de Lisboa.
}

\cvline{2021\\ \textbf{LNEC}}{
 Catarina Marques Graça, ``Validação do Modelo Numérico SWAN com base na
 comparação estatística de regimes de agitação marítima'', instituição de
 acolhimento: LNEC Portugal, classificação de 18 valores,  estágio curricular da Licenciatura em
 Matemática Aplicada à  Tecnologia e à Empresa (LMATE),
 orientação conjunta com Eng.$^{\underline{\mbox{a}}}$ Liliana Pinheiro
 (\textbf{LNEC}),
 Prof.$^{\underline{\mbox{a}}}$ Ana Filipa Prior (ISEL), março de 2021,  Departamento de Matemática do Instituto Superior de Engenharia de Lisboa.
}


\cvline{2020\\ \textbf{CDRSP-IPLeiria}}{Eurico Carnall Figueiredo, ``Uma Introdução aos Métodos
  Analíticos e Numéricos para a Simulação Computacional da Dinâmica de
  Fluidos:  Aplicação à Injecção Assistida por água em Moldes (WAIM)'',
 instituição de acolhimento: \emph{CDRSP}, classificação de 17 valores,  estágio curricular da Licenciatura em Matemática Aplicada à
  Tecnologia e à Empresa (LMATE),
  orientação conjunta com Prof.$^{\underline{\mbox{a}}}$ Doutora Paula Pascoal-Faria
  Faria (CDRSP),
  Prof. Doutor Ricardo Enguiça (ISEL),
  Prof. Doutor Tiago Charters de Azevedo (ISEL),   julho
  de 2020, área Departamental de Matemática do Instituto Superior de Engenharia de Lisboa.
}

\cvline{2020\\ \textbf{GALP}}{Diogo Joaquim,''Modelação Preditiva de Consumos no Mercado
  de Gás Natural Portugal e Espanha'', instituição de acolhimento:  \emph{GALP Portugal}, classificação de 18 valores,  estágio curricular da Licenciatura em Matemática Aplicada à
  Tecnologia e à Empresa (LMATE),
  orientação conjunta com Prof. Doutor Ricardo Enguiça (ISEL),
  Eng. Gonçalo Monteiro (GALP), julho de
  2020, área Departamental de Matemática do Instituto Superior de Engenharia de Lisboa.}


\cvline{2020 \\ \textbf{LNEC}}{Ana Rita Costa, ``Análise da operacionalidade
    do terminal portuário de S .Roque do Pico, com base no regime geral
    de agitação'', instituição de acolhimento: \emph{LNEC}, classificação de 16 valores,  estágio curricular da Licenciatura em Matemática Aplicada à
  Tecnologia e à Empresa (LMATE),
  orientação conjunta com Prof.$^{\underline{\mbox{a}}}$ Doutora Sandra
  Aleixo (ISEL), Eng.$^{\underline{\mbox{a}}}$ Juana Fortes (LNEC),
  Eng.$^{\underline{\mbox{a}}}$ Liliana Pinheiro (LNEC),  fevereiro
  de 2020, área Departamental de Matemática do Instituto Superior de Engenharia de Lisboa.}


\cvline{2019\\\textbf{GALP} }{Ruben Sales,''Otimização de Stocks e da Contratação e de
  Capacidade'', instituição de acolhimento:  \emph{GALP Portugal}, classificação de 16 valores,  estágio curricular da Licenciatura em Matemática Aplicada à
  Tecnologia e à Empresa (LMATE),
  orientação conjunta com Prof. Doutor Ricardo Enguiça (ISEL),
  Eng. Gonçalo Monteiro (GALP), julho de
  2019, área Departamental de Matemática do Instituto Superior de Engenharia de Lisboa.}

\cvline{2019 \\ \textbf{W{\"{u}}rth} Portugal}{Pedro Ferreira,  \emph{Otimização de preços e margens,
    W{\"{u}}rth Portugal}, classificação de 15 valores,  estágio curricular da Licenciatura em Matemática Aplicada à
  Tecnologia e à Empresa (LMATE),
  orientação conjunta com Prof. Doutor Tiago Charters de Azevedo (ISEL),
  Dr. Gonçalo Figueiredo (W{\"{u}rth} Portugal) e Dr. Jorge Fernandes
  (W{\"{u}rth} Portugal),  julho de 2019, área Departamental de Matemática do Instituto Superior de Engenharia de Lisboa.}


\section{Actividades  pedagógicas e organizacionais}
\subsection{Comissões e outros cargos de gestão}
\cvline{2018 -- 2022
  \phantom{a}[ISEL-IPL]
}
{ Membro da Comissão Coordenadora da Licenciatura
em Matemática Aplicada à Empresa e à Tecnologia (CCLMATE), de
junho de 2018 a maio de 2023.}

\cvline{2018 -- presente
  \phantom{a}[ISEL-IPL]
}
{Responsável por parte da LMATE pelo \emph{LAB4MAT}, laboratório
  computacional da LMATE, desde 17 de  outubro de 2018.}

\cvline{2018 [ISEL-IPL]}{Membro da Comissão Instaladora da Oficina
Digital (ODI), Instituto Superior de Engenharia de Lisboa, 2018.}

\subsection{Funções docentes desenvolvidas}
%% \cvline{2015 -- presente
%% \phantom{a}[ISEL-IPL]
%% }{Responsável por unidades curriculares:
%%   \begin{itemize}
%%   \item Cálculo Numérico da Licenciatura em Engenharia Química e Biológica;
%%     \item Introdução à Análise Numérica da Licenciatura em Matemática
%%       Aplicada à Tecnologia e à Empresa,
%%       \item Análise Vetorial da Licenciatura em Engenharia
%%         Eletrotécnica,
%%     \end{itemize}
%%     do Instituto Superior de Engenharia de Lisboa.}
\cvline{2002 -- presente \phantom{a}[ISEL-IPL]}{Leccionação (e/ou
  regência) em regime teórico-prático das
  unidades curriculares:
  \begin{itemize}
  \item Álgebra Linear e Geometria Analítica;
   \item Análise Matemática I;
   \item Análise Matemática II;
   \item Análise Vetorial;
   \item Análise Numérica;
   \item Cálculo Diferencial e Integral;
   \item Cálculo Diferencial e Integral  II;
   \item Cálculo Numérico;
   \item Complementos de Análise Matemática;
   \item Introdução à Análise Numérica;
   \item Matemática Aplicada à Engenharia Civil;
   \item Matemática Aplicada à Engenharia Electrotécnica,
   \end{itemize}
   no Instituto Superior de Engenharia de Lisboa do Instituto Politécnico de Lisboa.}

\cvline{2002 -- 2002 [ESTG-IPB]}{Leccionação em regime prático das
  unidades curriculares de
  Análise Matemática I e Álgebra Linear e Geometria Analítica, Escola
  Superior de Tecnologia e Gestão do Instituto Politécnico de Bragança.}

\cvline{2001 -- 2002 \phantom{a}[FCT-UNL]}{Leccionação em regime prático
  da unidade curricular de
  Análise Matemática I na Faculdade de Ciências e Tecnologia da
  Universidade Nova de Lisboa.}

\cvline{2000 -- 2001 [FC-UL]}{Leccionação em regime prático das unidades curriculares
  de Álgebra e Aplicações I e II na Faculdade de Ciências da Universidade
  de Lisboa.}


\subsection{Textos de apoio a unidades curriculares}

\cvline{2017}{N.~D.~Lopes, \emph{Métodos Numéricos: Notas Práticas}, Unidade Curricular de Introdução à Análise Numérica da
Licenciatura em Matemática Aplicada à Tecnologia e à Empresa da
  área Departamental de Matemática do Instituto Superior de Engenharia de Lisboa.}


\cvline{2016}{N.~D.~Lopes, \emph{Introduction to Numerical Calculus with
  Python}, Unidade Curricular de Cálculo Numérico da Licenciatura em Engenharia Química e Biológica da
  área Departamental de Engenharia Química do Instituto Superior de Engenharia de Lisboa.}

\cvline{2016}{N.~D.~Lopes, R. Enguiça, I. Coelho e P. Simões, \emph{Funções
    Elementares -I },  Unidade Curricular de
  Funções Elementares - I do curso de preparação  ''Maiores de 23''
  leccionado pela área Departamental de Matemática do Instituto Superior de Engenharia de Lisboa.}

\cvline{2015}{N.~D.~Lopes, \emph{Introdução ao Cálculo Numérico com
    Python}, Unidade Curricular de Cálculo Numérico da Licenciatura em Engenharia Química e Biológica da
  área Departamental de Engenharia Química do Instituto Superior de Engenharia de Lisboa.}


\subsection{Júri}
\cvline{2024}{Vogal (arguente) de júri das provas de Mestrado de Cristian Robu. Relatório de Estágio de Natureza Profissional
realizado na “Infraestruturas de Portugal”, intitulado “The Use of Artificial Intelligence in the
Recognition of Railway Assets Based on High-Resolution Drone Images”, classificação de 18 (dezoito) valores.}

\cvline{2024}{Vogal de júri de concurso $\# 4603$ para atribuição de duas Bolsas de Iniciação à Investigação (BII) na FCiências.ID - Associação Para a Investigação e Desenvolvimento de Ciências, no âmbito do projeto “UIDP/04621/2020”, 
\url{doi:10.54499/UIDP/04621/2020".}}

\cvline{2023}{Vogal de júri de bolsa de Doutoramento na área de
  Digitalização e Inteligência Artificial -- Informática/Matemática,
  atribuída ao abrigo do Protocolo de Colaboração para Financiamento do Plano Plurianual de Bolsas de Investigação para Estudantes de
  Doutoramento, celebrado entre a FCT e o IPS. Edital de 25 de janeiro
  de 2023, IPS}

\cvline{2022}{Vogal de júri de concurso  para atribuição de uma Bolsa de Iniciação à Investigação,
  referência \url{IPL/2022/3A\_ISEL/BII/5M}, no âmbito do Projeto de Investigação Científica e Desenvolvimento
Tecnológico designado por 3A-AdaptiveAllocApp, financiado na íntegra por fundo do Instituto Politécnico de Lisboa.}

\cvline{2022}{Presidente de júri de concurso  para atribuição de uma Bolsa de Iniciação à Investigação, referência \url{IPL/2022/3A\_ISEL/BII/4M}, no âmbito do Projeto de Investigação Científica e Desenvolvimento
Tecnológico designado por 3A-AdaptiveAllocApp, financiado na íntegra por fundo do Instituto
Politécnico de Lisboa.}

  \cvline{2022}{Presidente de júri das provas de defesa de estágio
  curricular de Maria Silva,   estágio LMATE-- \emph{Milestone}, 26 de julho
  de 2022, área Departamental de Matemática do Instituto Superior de Engenharia de Lisboa.}


  \cvline{2021}{Arguente  das provas de defesa de estágio
  curricular de  Ricardo Dias,  estágio LMATE--\emph{Arquiled}, julho
  de 2021, área Departamental de Matemática do Instituto Superior de Engenharia de Lisboa.}

  \cvline{2020}{Presidente de júri das provas de defesa de estágio
  curricular de  Carolina Figueiredo,   estágio LMATE--AdTA, 13 de fevereiro
  de 2020, área Departamental de Matemática do Instituto Superior de Engenharia de Lisboa.}


\cvline{2020}{Presidente de júri das provas de defesa de estágio
  curricular de  Catarina Ferreira,  estágio LMATE--CML, 11 de fevereiro
  de 2020, área Departamental de Matemática do Instituto Superior de Engenharia de Lisboa.}


\cvline{2019}{Presidente de júri das provas de defesa de estágio
  curricular de Rafaela Couchinho,  estágio LMATE--GALP, 9 de julho
  de 2019, área Departamental de Matemática do Instituto Superior de Engenharia de Lisboa.}


%% \subsection{Orientações de trabalhos finais de Mestrado}
%% \cvline{2016}{Gonçalo Galego,  {ciclo de estudos  não concluído},
%%   orientação conjunta com Prof. Doutor Tiago Charters de Azevedo,
%% "Optimização e modelação paramétrica de turbinas de vento verticais:
%% uma abordagem computacional com recurso técnicas de modelação 3D
%%  ", área Departamental de Mecânica  do Instituto Superior de Engenharia de Lisboa.}


%% \subsection{Alunos em regime de tutorial}
%% \cvline{2020}{Eurico Carnall Figueiredo, Unidade Curricular de Temas da
%%   Matemática da Licenciatura em Matemática Aplicada à Tecnologia e à
%%   Empresa, com relatório final: ``Uma Introdução aos Métodos
%%   Analíticos e Numéricos para a Simulação Computacional da Dinâmica de
%%   Fluidos:  Aplicação à Injecção Assistida por água em Moldes'', 18
%%   valores. Júri Prof.$^{\underline{\mbox{a}}}$ Doutora Sandra Aleixo, Prof.$^{\underline{\mbox{a}}}$ Doutora Lucía Suarez,
%%   19 de fevereiro de 2020, área
%%   Departamental de Matemática do Instituto Superior de Engenharia de
%%   Lisboa.}

%% \cvline{2017}{Irem Toksoz, Unidade Curricular de Cálculo Numérico da Licenciatura em Engenharia Química e Biológica da
%%   área Departamental de Engenharia Química do Instituto Superior de Engenharia de Lisboa.}
%% \cvline{2016}{Mohammed Alabdali e Altona Aktas, Unidade Curricular de Cálculo Numérico da Licenciatura em Engenharia Quimica e Biológica da
%%   área Departamental de Engenharia Química do Instituto Superior de Engenharia de Lisboa.}

\subsection{Actividades de promoção e divulgação científica}
\cvline{2018}{N. D. Lopes, ``Ondas, Poluentes e Turbilhões'',
  comunicações  na Semana Aberta do  Instituto Superior de Engenharia de
  Lisboa, 20 a 23 de fevereiro de 2018}

\cvline{2017}{N. D. Lopes e N. Santos, ``A Matemática no Surf de Ondas
  Grandes -- Previsão e Modelação '',  palestra convidada na  Escola de
  Verão da Sociedade Portuguesa de Matemática $\&$ Mat\-Oeste 2017 --
  Matemática do Mar, ESTG, Instituto Politécnico de Leiria, 13 de julho
  de 2017.}

\cvline{2017}{N. D. Lopes, ``Alguns Modelos Analíticos e Numéricos para
  a Geração e Propagação de Ondas Marítimas de Superfície'', comunicação
  convidada na palestra ``Nossos Oceanos, Nosso Futuro'', Instituto
  Superior de Engenharia de Lisboa, 22 de junho de 2017.}

\cvline{2016}{N. D. Lopes, ``Introdução ao Python para a LMATE'',
  seminário no curso de Licenciatura em Matemática Aplicada à
  Tecnologia e à Empresa, Instituto Superior de Engenharia de Lisboa, 12 de outubro de 2016.}

\cvline{2015}{N.D. Lopes, ``Algumas Simulações Numéricas com o
  Método dos Elementos Finitos: Convecção-Difusão; Boussinesq;
  Navier-Stokes'', aula na unidade curricular de Mecânica de Fluidos Computacional, do
  Curso de Mecânica do Instituto Superior de Engenharia de
  Lisboa, por convite do Prof. Doutor Pedro Patrício, 10 de dezembro de 2015.}

\cvline{2015}{Colaboração no stand da área Departamental de Matemática
  na ``Feira da Ciência e Tecnologia no ISEL''
  com a exposição de simulações numéricas resultantes do trabalho de
  investigação em modelos de Mecânica de Fluidos, 25 de Novembro de 2015.}

\cvline{2015}{Colaboração no stand ``DIY'', da responsabilidade do
  Prof. Doutor Tiago Charters de Azevedo, na  ``Feira da Ciência e Tecnologia no
  ISEL'' com a exposição de um sintetizador de audio, 25 de Novembro de 2015.}

\cvline{2015, 2016}{Colaboração como representante da área
  Departamental da Matemática do Instituto Superior de Engenharia de Lisboa nas edições de 2015 e 2016 da
  feira ``Futurália'', Lisboa.}

\cvline{2010, 2011, 2012}{N.~D.~Lopes, ``Ondas, poluentes e
  turbilhões'', seminário de divulgação científica inserido nos eventos: \emph{MatNova 2012},
  \emph{Matemática na FCT 2011} e \emph{ExpoFCT 2010}, Faculdade de
  Ciências e Tecnologia da Universidade Nova de Lisboa.}

\cvline{2004, 2005}{N.~D.~Lopes, ``Análise Numérica de um Modelo de
  Convecção-Difusão: Aplicação ao Estuário do Tejo'', seminário de divulgação científica inserido nos cursos
  \emph{Matemática Alpha 2004} e \emph{Matemática Alpha 2005}, Faculdade
  de Ciências da Universidade de Lisboa.}

\section{Outras Informações}
\cvline{}{Linguagens de Programação: Python, Julia, C++.}
\cvline{}{Línguas: inglês (bons níveis de leitura, escrita e
  conversação).}


\end{document}
